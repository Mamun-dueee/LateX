\documentclass{article}
\usepackage{graphicx}
\usepackage[utf8]{inputenc}
\usepackage[rightcaption]{sidecap}
\usepackage{listings}

\usepackage{wrapfig}

\title{\textbf{Microcontroller Based Bidirectional Digital Visitor Counter}}
\author{Md. Al Mamun Siddiki, SH-81, 3rd year, 2015-2016,\\ Department of EEE,University of Dhaka. }
\date{16th October 2016}

\begin{document}

\maketitle

\section{Abstract}
This paper presents the design and construction of a digital bidirectional visitor counter (DBVC). The DBVC is a
reliable circuit that takes over the task of counting number of persons / visitors in the room very accurately and
beeps a warning alarm when the number of visitors exceeds the capacity limit of the auditorium/hall. When
somebody enters the room then the counter is incremented by one (+1) and when any one leaves the room then
the counter is decremented by one (-1). The total number of persons inside the room is also displayed on the
LCD (Liquid Crystal Display). 
\\
The microcontroller is used for detecting an entry or exit action and computing the figures (addition and
subtraction) to acquire accurate results. It receives the signals from the sensors, and this signal is operated under
the control of embedded programming code which is stored in ROM of the microcontroller. The microcontroller
continuously monitors the Infrared Receivers. When any object pass through the IR Receiver's then the IR Rays
falling on the receivers are obstructed.
\\
The obstruction occurs under two circumstances, either you obstruct sensor 1 (i.e. outside the building) before
sensor 2 (i.e. which is inside the building) this shows that you are entering the building or you do it the other way
round, which is obstructing sensor 2 before sensor 1 to indicates an exit movement. This obstruction is sensed by
the Microcontroller, computed and displayed by a 16x2 LCD screen.

\section{Introduction}
Visitor counting is simply a measurement of the visitor traffic entering and exiting conference rooms, malls,
sports venues, etc. With the increase in standard of living, there is a sense of urgency for developing circuits that
would ease the complexity of life.

\par
Over the years, the usage of Visitor counters has become very positive in terms of monitoring crowd behavior at
a particular place. It began with a mechanical tally counter which was introduced to replace the use of tally stick.
A tally (or tally stick) was an ancient memory aid device used to record and document numbers, quantities, or
even messages. Historical reference is made by Pliny the Elder (AD 23–79) about the best wood to use for
tallies, and by Marco Polo (1254–1324) who mentions the use of the tally in China. Tallies have been used for
numerous purposes such as messaging and scheduling, and especially in people counting, financial and legal
transactions, to the point of being accuracy.\\
\par

The substitute of the tally stick was the mechanical tally counter, it is a device used to incrementally count
something, typically passing. One of the most common things tally counters are used for is counting people,
animals, or things that are quickly entering and existing a location.\\

\par
As times went on, an electronic tally counter was introduced which used an LCD screen to display the count, and
a push button to advance the count. Some also have a button to decrement the count in case of a miscount. Now,
due to technology advancement, various type of people counter has been introduced to automatically count the
number of people entering and exiting a building at a particular time. Some of these are laser beam, thermal
imaging, video camera and the infra-red sensor. All these sensors play their role respectively as visitor detector.
These devices are very reliable and accurate in terms of performance as compared to the mechanical tally
counter.\\

\par
In the past years, several well established institutions (libraries, community centers, auditorium, etc.) across the
globe have encountered various incidents related to traffic monitoring. It has been a necessity to monitor the
visitors to carry out the human traffic management task and tourist flow estimate to vindicate accurate result for
the organizational marketing and statistical research. This eventually indicates the patronage rate of goods and
services by consumers. Therefore, we deem it appropriate to identify these problems encountered by our various
organizations and find solutions to them by designing a digital bidirectional visitor counter (DBVC).\\

\par
The primary method for counting the visitors involves hiring human auditors to stand and
number of visitors who enter or pass by a certain location.\\

\par
manually tally the
The human auditing application or the human-based data collection was unreliable and came at great cost. For
instance, in situations where a large number of visitors entering and exiting buildings such as conference rooms,
law courts, libraries, malls and sports venues, going for human auditors to manually tally the number of visitors
may result in inaccurate data collection. For this reason, many organizations have tried to find solutions to
mitigate the inaccurate traffic monitoring issues. It is our intention to design and construct this digital
bidirectional visitor counter (DBVC) with maximum efficiency and make it very feasible for anyone who wants
to design and construct the prototype. Building this circuit will provide information to management on the
volume and flow of people in a building.\\

\par
Our main objective in this paper includes designing and constructing a visitor counter which will make a
controller based model to count and compute the number of visitors in a building at a particular time. It is also
our objective that this controller base model beeps a warning alarm when the capacity of the building is
exceeded.\\

\par
The significance of the design and construction in this paper is enshrined in the fact that it provides the assurance
of the health and safety of the occupants in a building at all time, since the visitors are guaranteed of traffic
decongestion. It also provides accurate data for various research and analytical purposes as it generates the
hourly, daily, monthly, and yearly report. The device helps to reduces pressure on building facilities by
prompting the security, when the capacity of the building is exceeded. It goes a long way to assist rescue team or
security services to come up with strategic procedure in dealing with emergency issues like people trapped in a
structure as a result of hijacks and collapsed building which occurred recently at the West End Gate Mall in
Kenya and Melcom in Ghana respectively.\\
It is the usual norm that the design and construction of every device comes with some limitations and ours
cannot be an exception. In this paper, our device might count more than two people as one when they interrupt
the infrared beam at the same time in a linear direction. For this reason, the device must be installed at a narrow
entrance/exit where one person enters at a time. Another limitation can be linked to the inability of sensor in the
device to differentiate between human being and objects interrupting the IR signal. Finally, the device will fail to
function in case of any power interruption, which might lead to a miscount or provide inaccurate data when
power is restored.

\section{Methodology}
This section introduces the methodology involved in the design and construction of the Digital Bidirectional
Visitor Counter (DBVC). Using the Takoradi Polytechnic Library crowd management situation as a case study,
it was realized that the library’s capacity often gets exceeded during its peak usage period (examination period)
and therefore makes the environment uncomfortable for learning. This problem was studied by visually
observing students reaction anytime the library’s capacity was exceeded. Another study was made on the
Melcom tragedy incident, whereby the exact number of people trapped in the collapse building was unknown.
False information about the number of people trapped was given to the rescue team at their arrival, but they
ended up rescuing more survivors than the expected number revealed to them. This means a lot of people could
have died if the rescue team relied on the information given to them.\\

\par
This chapter covers all parts of a DBVC from the system overview to the individual components required to
assemble the visitor counter to provide effective crowd management as in monitoring and controlling. The
microcontroller based visitor counter is designed to respond to the flaws in the operations of the existing
counters. The design in its sense has four (4) main sections and circuits as shown in Figure 1. These include
detection section (IR sensor circuitry), microcontroller section, alerting section (LCD and Buzzer) and power
supply circuit.

\subsection{Microcontroller Section}
The microcontroller section consists of the PIC16F877A Microcontroller which is a powerful (200 nanosecond
instruction execution) easy-to-program (only 35 single word instructions) CMOS FLASH-based 8-bit
microcontroller packs Microchip's powerful PIC architecture into a 40-pin package and is upwards compatible
with the PIC12CXXX and PIC16C7X devices. The PIC16F877A features 256 bytes of EEPROM data memory,
self-programming, an ICD, 2 Comparators, 8 channels of 10-bit Analog-to-Digital (A/D) converter.
    

\vspace{2cm}
\begin{SCfigure}[0.5][h]
\caption{Pictorial view of a PIC16F877A microcontroller}
\includegraphics[width=0.6\textwidth]{PIC16F877A}
\end{SCfigure}

\newpage
\subsection{Circuit Diagram}
I use LED instead of LCD monitor and  push button instead of IR sensors. 
\vspace{1.5cm}
\includegraphics[width=11cm,height=11cm]{circuit}

\subsection{Program(for HI-TECH C compiler)}
\begin{lstlisting}
#include <pic.h>
#define _XTAL_FREQ 20e6
__CONFIG(0x3F3A);
unsigned char count=0;
main()
{
ADCON1=0b00001111;    //to make PORTE as digital I/O pins.
TRISD=0;
TRISE0=1;
TRISE1=1;
while(1)
    {
    if(RE0==1)
        {
        if(count<255){count++;}
        __delay_ms(50);
        while(RE0==1);
        __delay_ms(50);
        }
    
    if(RE1==1)
        {
        if(count>0){count--;}
        __delay_ms(50); 
        while(RE1==1);
        __delay_ms(50);
        }
    PORTD=count;
    }
}
\end{lstlisting}

\section{Results and Discussion}
The actual project was very defficult to implement. So for simplicity I used push button instead of IR sensors and also used LED for display the results instead using a LCD monitor. For entering os a person I use the 1st push button and existing of a person I use 2nd push button.The result is displayed in binary form by LEDs. 

\section*{References}
[1] blog.vinu.co.in 

\end{document}
